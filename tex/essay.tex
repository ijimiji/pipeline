\begin{center}
  \large\bfseries{РЕФЕРАТ}
\end{center}

Дипломная работа, 54 стр., 10 источников.

\textbf{Ключевые слова:} микросервисы, приложения с нейронными сетями, разработка программного обеспечения.

\textbf{Объекты исследования --} методы разработки масштабируемых приложений для эксплуатации нейронных сетей.

\textbf{Цель исследования --} систематизировать подходы к разработке микросервисов, выделить и применить ключевые методы разработки для приложений с нейронными сетями.

\textbf{Методы исследования --} изучение соответствующей литературы и электронных источников, постановка задачи и её решение.

\textbf{В результате исследования --} раскрыты и графически проиллюстрированы основные понятия микросервисной разработки, разработано приложение для генерации изображений.

\textbf{Области применения --} компьютерные технологии.

\newpage

\begin{center}
  \large\bfseries{ABSTRACT}
\end{center}

Thesis, 54 pages, 10 sources.

\textbf{Keywords:} microservices, neural network applications, software development.

\textbf{Objects of research --} methods for developing scalable applications for the implementation of neural networks.

\textbf{Aim of the study --} to systematize approaches to the development of microservices, identify and apply key development methods for neural network applications.

\textbf{Methods of research --} study of relevant literature and electronic sources, problem setting and its solution.

\textbf{Results of the study --} revealed and graphically illustrated the main concepts of microservice development, developed an application for image generation.

\textbf{Fields of application --} computer technologies.

\newpage

\begin{center}
  \large\bfseries{РЭФЕРАТ}
\end{center}

Дыпломная праца, 54 старонкі, 10 крыніц.

\textbf{Ключавыя словы:} мікрасервісы, прылады з нейроннымі сеткамі, распрацоўка праграмнага забеспячэння.

\textbf{Аб'екты даследвання --} метады распрацоўкі маштабуемых прылад для эксплуатацыі нейронных сетак.

\textbf{Мэта даследвання --} сістэматызацыя падыходаў да распрацоўкі мікрасервісаў, вылучэнне і прымяненне асноўных метадаў распрацоўкі для прылад з нейроннымі сеткамі.

\textbf{Метады даследвання --} вывучэнне адпаведнай літаратуры і электронных крыніц, пастаноўка задачы і яе рашэнне.

\textbf{Вынікі даследвання --} раскрыты і графічна ілюстраваны асноўныя паняцці мікрасервіснай распрацоўкі, распрацавана прылада для генерацыі малюнкаў.

\textbf{Сферы выкарыстання --} камп'ютэрныя тэхналогіі.

\newpage