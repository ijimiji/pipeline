\chapter*{\large ВВЕДЕНИЕ}  
\addcontentsline{toc}{chapter}{ВВЕДЕНИЕ}
В последние годы микросервисная архитектура значительно приобретает популярность в области разработки программного обеспечения. Основная идея микросервисов заключается в разбиении сложных программных систем на отдельные, слабо связанные компоненты с целью обеспечения гибкости, упрощения процессов разработки и масштабирования. Однако, эффективное управление такими многочисленными и взаимодействующими сервисами предполагает наличие централизованного описания и систематизированного подхода к их разработке и поддержке.

Целью данной работы является разработка масштабируемой и надежной 
архитектуры для инференса изображений с помощью нейронных сетей.

В процессе исследования рассмотрены существующие методы, инструменты и подходы, применяемые для проектирования микросервисных архитектур и разработана система, удовлетворяющая критериям микросервисной
архитектуры для генерации изображений из пользовательских запросов.

