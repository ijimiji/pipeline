\chapter*{\large ВВЕДЕНИЕ}  
\addcontentsline{toc}{chapter}{ВВЕДЕНИЕ}
В последние годы микросервисная архитектура значительно приобретает популярность в области разработки программного обеспечения.
Основная идея микросервисов заключается в разбиении сложных программных систем на отдельные,
слабо связанные компоненты с целью обеспечения гибкости, упрощения процессов разработки и масштабирования.
однако, эффективное управление такими многочисленными и взаимодействующими сервисами предполагает 
наличие централизованного описания и систематизированного подхода к их разработке и поддержке.

Наравне с развитием инструментов разработки программного обеспечения для веб-приложений 
широкое распространение получили также технологии, связанные с глубоким обучением. Нейронные сети могут решать широкий спектр
задач, но требуют значительного количества вычислительных ресурсов, что ставит перед разработчиками задачу построения
отказоустойчивых и масштабируемых систем, которые обрабатывают запросы для нейронных сетей.

Целью данной работы является описание существующих подходов к разработке микросервисных архитектур приложений, разработка 
масштабируемой и надежной архитектуры для инференса изображений с помощью нейронных сетей, описание трудностей, которые
возникают в процессе разработки и обозначение возможных путей решения подобных проблем.

В процессе исследования рассмотрены существующие методы, инструменты и подходы,
применяемые для проектирования микросервисных архитектур, приведены архитектуры нейронных сетей, которые используются для генерации
изображений, и разработана система, удовлетворяющая критериям микросервисной
архитектуры для генерации изображений из пользовательских запросов.

