\section{Основные результаты}
В данной работе был рассмотрен подход к разработке программного обеспечения на основе микросервисной архитектуры с использованием декларативного описания и протокола gRPC для взаимодействия между микросервисами. Совокупность данных подходов и инструментов позволяет повысить уровень абстракции при разработке, упростить и ускорить процесс кодирования и обеспечить лучшее взаимодействие с инфраструктурными, вычислительными и сетевыми ресурсами.

Основной идеей данной работы является формальное описание связей и хостов микросервисов в виде JSON-структуры. Отсюда мы приступили к использованию графа для управления запросами от и к микросервисам и обработке вызовов этих сервисов из главного приложения. Используя этот подход, мы показали возможность гибкости и модульности в разработке: описание индивидуальных сервисов не требует изучения сложного кода всего проекта.

Для обеспечения быстрого и эффективного обмена данными между микросервисами, управляемыми основным приложением, были использованы gRPC и Protocol Buffers. Это позволяет быстро и надежно распределять запросы и ответы между микросервисами, вызывая их по мере необходимости. 

Внедрение подобного подхода может упростить разработку и ускорить процесс кодирования при использовании микросервисной архитектуры благодаря декларативному описанию структуры приложения:
\begin{itemize}
    \item Легкость просмотра и изменения описания микросервисов и их зависимостей, так как все отношения отображаются в виде отдельного JSON файла.
    \item Упрощение изменения и обновления микросервисов, так как достаточно изменить JSON файл для изменения конфигурации приложения.
    \item Облегчение командной работы над проектом и координации между разработчиками, так как каждая команда может работать над своим микросервисом, оперируя только его описание и без необходимости полного погружения в код всего приложения.
\end{itemize}

Использование декларативного описания позволяет улучшить процесс управления запросами и передачей данных между микросервисами с использованием графа:
\begin{itemize}
    \item Благодаря графовой логике основному приложению необходимо лишь знать порядок вызова и зависимости между узлами графа, что позволяет упростить и сделать более надежным обработку запросов.
    \item Гарантия корректной передачи данных между микросервисами, которые представлены узлами графа.
    \item Возможности использования различных алгоритмов обработки графов, например топологической сортировки или алгоритма обходов графа (DFS, BFS), для повышения гибкости системы.
\end{itemize}


Использование протокола gRPC и Protocol Buffers также позволяет достичь эффективного обмена данными между микросервисами с помощью бинарного протокола передачи данных и сериализации, как и Облегчить процесс масштабирования и распределения приложений на разных серверах или виртуальных машинах, так как gRPC предоставляет простой и современный подход к удаленным вызовам процедур (RPC).

Поддержка множества языков программирования и платформ, что упрощает процесс интеграции с другими системами и позволяет командам разработчиков использовать предпочтительные инструменты и технологии.

Однако следует также отметить потенциальные недостатки и ограничения данного подхода:
\begin{enumerate}
    \item Система должна быть способна корректно обрабатывать ошибки и предоставлять информацию об ошибках в случаях неудачного вызова одного из микросервисов, что может потребовать дополнительного управления ошибками и внедрения средств мониторинга.
    \item Может возникнуть проблема с производительностью из-за обращений и вызовов микросервисов на каждый входящий запрос. Увеличение числа вызовов микросервисов может привести к задержкам и увеличению времени ответа. В этом случае требуется оптимизировать реализацию графа и возможно использовать кэширование для ускорения процесса.
\end{enumerate}


Таким образом, использование декларативного описания и микросервисной архитектуры с графовым подходом к управлению запросами и передачей данных предлагает возможности для повышения модульности, гибкости и масштабируемости разрабатываемых приложений. Выводы, сделанные на основе реализации подобного подхода, указывают на продуктивность и разнообразные возможности его использования, при условии должного учета потенциальных проблем и ограничений. Несмотря на некоторые вызовы в области производительности и сложности, интеграция декларативного описания, графов и протокола gRPC может оказаться ценным инструментом для современных разработчиков программного обеспечения.