\chapter*{ \large ЗАКЛЮЧЕНИЕ}
\addcontentsline{toc}{chapter}{ЗАКЛЮЧЕНИЕ}

Микросервисная методология позволяет решать ряд проблем связанных с масштабированием и отказоустойчивостью систем, однако налагает ряд ограничений и имеет ряд недостатков, с которыми приходится сталкиваться в процессе разработки. Некоторые из данных проблем предлагается решить путем переноса концепции наличия контракта формата передачи данных на взаимодействие между сервисами через введение понятия декларативной микросервисной архитектуры.

В ходе работы было рассмотрено понятие микросервисной архитектуры и произведен обзор имеющихся средств и методологий разработки, применяющихся для коммуникации веб-сервисов.

Результатом работы стала разработка программного обеспечения для генерации изображений
с помощью нейронной сети с сетевым интерфейсом. 
Рассмотрены подходы общения разных процессов и проанализированы достоинства и недостатки каждого из методов.
