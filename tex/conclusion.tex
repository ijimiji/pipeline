\chapter*{ \large ЗАКЛЮЧЕНИЕ}
\addcontentsline{toc}{chapter}{ЗАКЛЮЧЕНИЕ}

Микросервисная разработка заключается в построении распределенных систем, которые удовлетворяют
критериями масштабируемости и низкой связности отдельных ее узлов.
Она позволяет решать ряд проблем связанных с масштабированием и отказоустойчивостью систем, 
однако налагает ряд ограничений и имеет ряд недостатков, 
которыми приходится сталкиваться в процессе разработки.
Для решения подобных проблем большое внимание уделяется инструментации приложений, описанию контрактов
передачи данных, использованию сетевых и прикладных протоколов, которые обеспечивают гарантии
консистентности данных в зависимости от предъявляемых требований к производительности.

Отдельное внимание также уделяется вопросами инфраструктуры: наличию нескольких зон доступности, 
применению практик непрерывной интеграции, тестирования и использования баз данных и брокеров сообщений,
которые можно горизонтально масштабировать вместе с приложением.

При выполнению предписываемых рекомендаций по внедрению микросервисных архитектур получается
добиться горизонтального масштабирования приложений, распределения сложности системы 
между ее независимыми узлами и выдерживать периоды повышенной нагрузки.
В ходе работы было рассмотрено понятие микросервисной архитектуры и произведен обзор имеющихся средств и подходов разработки,
применяющихся для коммуникации веб-сервисов.
В данной работе также было показано, что микросервисная разработка хорошо подходит для обработки запросов
по генерации изображений, позволяет независимо масштабировать разные виды вычислительных ресурсов и 
позволяет более надежно обрабатывать большое количество запросов при ограниченных ресурсах GPU.

Результатом работы стала разработка программного обеспечения для генерации изображений
с помощью нейронной сети с сетевым интерфейсом, которое представляет собой систему из трех микросервисов,
демонстрирующих различные виды синхронного и асинхронного сетевого взаимодействия, а также технологий,
которые применяются при микросервисной разработке. 

На примере данной системы также рассмотрены проблемы, которые могут возникнуть при технических неполадках
и предложены варианты их решения.

