\chapter*{ \large ЗАКЛЮЧЕНИЕ}
\addcontentsline{toc}{chapter}{ЗАКЛЮЧЕНИЕ}

Микросервисная архитектура позволяет решать ряд проблем связанных с масштабированием и отказоустойчивостью систем, 
однако налагает ряд ограничений и имеет ряд недостатков, которыми приходится сталкиваться в процессе разработки.

В данной работе было показано, что паттерны разработки микросервисов хорошо ложатся на задачи, связанные
с обработкой запросов, требующих большого количества вычислительных ресурсов, в частности, генерации
изображений с помощью нейронных сетей.

Результатом работы стала разработка программного обеспечения для генерации изображений
с помощью нейронной сети с сетевым интерфейсом, которое представляет собой систему из трех микросервисов,
демонстрирующих различные виды синхронного и асинхронного сетевого взаимодействия, а также технологий,
которые применяются при микросервисной разработке. 

На примере данного приложения была показана ключевая роль брокеров сообщений в обработке запросов,
координации работы отдельных сервисов, а также поддержании консистентности данных при наличии 
распределенных транзакций.

Кроме того, в данной работе были описаны механизмы передачи бинарных данных крупного размера через
брокеры сообщений с применением объектных хранилищ, а также механизмы отмены сообщений с помощью
NoSQL баз данных типа ключ-значение, что делает описанные подходы к реализации обработки запросов
независимыми от типа выбранного брокера сообщений и его системных ограничений.

Анализ возможных микросервисовных архитектур для обработки запросов нейронными сетями показал,
что наличие отдельного сервиса, осуществляющего инференс через обученную модель, позволяет 
масштабировать остальные сервисы даже при нехватке GPU ресурсов.

Опыт реализации описанной архитектуры приложения может принести пользу при проектировании подобных систем
и систематизирует набор практик, необходимых при применении паттернов разработки микросервисов.

