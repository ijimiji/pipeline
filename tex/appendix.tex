\titleformat{\section}[block]
  {\large\bfseries\centering}
  {\thesection\ }{}{}

\titleformat{\chapter}[display]{\normalfont\bfseries\raggedleft}{\chaptertitlename\ \thechapter}{18pt}{\Large}

\chapter*{ПРИЛОЖЕНИЕ А}
\addcontentsline{toc}{chapter}{ПРИЛОЖЕНИЕ А}
\section*{РЕПОЗИТОРИЙ ПРИЛОЖЕНИЯ НА GITHUB}

Код приложения находится в открытом доступе на платформе коллаборации GitHub по ссылке, представленной в виде QR-кода.
Практическая часть, упоминаемая в основной работе, представлена пакетами cmd и internal. Также присутствует docker-compose файлы необходимые для поднятия
нужного окружения с томами для брокера сообщений, базы данных, хранилища медиа-файлов и инструментации приложения.

\begin{footnotesize}
\begin{figure}[h]
  \centering
  \includegraphics[width=0.89\textwidth]{img/frame.png}
  \caption*{}
\end{figure}

\end{footnotesize}

Далее будут приведены ключевые файлы конфигурации Docker контейнеров, необходимых для запуска приложения,
описания API, по которому сервис-шлюз коммуницирует с основным сервисом генерации, в формате Proto\-buf.

\section*{DOCKER COMPOSE СПЕЦИФИКАЦИЯ ДЛЯ ЛОКАЛЬНОГО ЗАПУСКА МИКРОСЕРВИСОВ}
\begin{lstlisting}[]
version: "3.8"

services:
  localstack:
    container_name: "${LOCALSTACK_DOCKER_NAME:-localstack-main}"
    image: localstack/localstack
    ports:
      - "127.0.0.1:4566:4566"
      - "127.0.0.1:4510-4559:4510-4559"
    environment:
      - DEBUG=${DEBUG:-0}
    volumes:
      - "${LOCALSTACK_VOLUME_DIR:-./volume}:/var/lib/localstack"
      - "/var/run/docker.sock:/var/run/docker.sock"

  prometheus:
    image: prom/prometheus:v2.25.0
    volumes:
      - ./prometheus:/etc/prometheus
      - prom_data:/prometheus
    command:
      - '--config.file=/etc/prometheus/prometheus.yml'
    ports:
      - 9090:9090

  grafana:
    image: grafana/grafana:7.5.7
    ports:
      - 3000:3000
    restart: unless-stopped
    volumes:
      - grafana-data:/var/lib/grafana
      - ./grafana:/etc/grafana/provisioning/datasources

  jaeger:
    image: jaegertracing/all-in-one:latest
    environment:
      - COLLECTOR_OTLP_ENABLED=true
    ports:
      - 16686:16686
      - 4318:4318
      - 5778:5778

volumes:
  grafana-data:
  prom_data:
\end{lstlisting}


\section*{GRPC СПЕЦИФИКАЦИЯ СЕРВИСА ГЕНЕРАЦИИ}
\begin{lstlisting}[]
  syntax = "proto3";

package proto.echo;

option go_package = "./proto";

service Core {
  rpc Generate(GenerateRequest) returns (GenerateResponse);
  rpc Discard(DiscardRequest) returns (DiscardResponse);
  rpc Status(StatusRequest) returns (StatusResponse);
}

message GenerateRequest {
  string Prompt = 1;
}

message GenerateResponse {
  string ID = 2;
}

message StatusRequest {
  string ID = 1;
}

message StatusResponse {
  ImageGroup ImageGroup = 1;
}

message DiscardRequest {
  string ID = 1;
}

message DiscardResponse {
}

message Image {
  string ID = 1;
  string Prompt = 2;
  string URL = 3;
  string Status = 4;
}

message ImageGroup {
  string ID = 1;
  repeated Image Images = 2;
}
\end{lstlisting}

