\section{Обзор языка программирования Golang }

Go (также известный как Golang) является статически типизированным компилируемым языком программирования, разработанным на замену ранним языкам системного программирования, таким как C\textsuperscript{++} и Java. Созданный Google, Go предоставляет упрощенный синтаксис и мощные инструменты для разработки операционных систем, сетевых сервисов, микросервисов и других распределенных систем \cite{golang}.

Для простоты внедрения данного языка было уделено большое внимание простоте синтаксиса и грамматике языка. Многие задачи имеют лишь одно решение средствами языка Go, которое описывается общеизвестными методами парадигм ООП и структурного программирования. За счет просты грамматики также достигается высокая скорость компиляции, что положительно сказывается на опыте разработки.

Также для устранения факторов, мешающих разработке, в стандартной поставке языка также предусмотрены инструменты для управления зависимостями и сборки, форматирования кода, синтаксического анализа, генерации и просмотра документации, а также профилирования готовых приложений.

Несмотря на то, что исходные коды языка находятся в общем доступе под свободной лицензией, процесс разработки по большей части контролируется разработчиками Google и многие решения проходят процесс согласования с проектировщиками языка в лице Кена Томпсона, который участвовал в разработке системы UNIX, Роба Пайка, известного за вклад в развитие операционной системы Plan9, и Роберта Гриземера, который до этого работал над виртуальной машиной V8 для языка JavaScript.

Поэтому, с одной стороны, участие сторонних разработчиков над ядром языка ограничено, поэтому новый функционал принимается с большим количеством обсуждений и согласований, за что язык регулярно критикуют.

Есть мнение, что в текущей реализации языка отрицается многолетний опыт разработки других языков, что отражается в использовании очень простой системы типов, сборщика мусора на поколенческом алгоритме и общей невыразительности базовых синтаксических элементов языка .

С другой стороны, подобный подход гарантирует полную обратную совместимость со старыми программами и маленькое ядро языка с выверенным набором инструментов, которые призваны уменьшать количество дискуссий по поводу вопросов, которые не имеют к разработке прямого отношения: сборка, форматирование, дистрибуция. Что в связке с богатой библиотекой для работы с сетевыми стеками делает Golang хорошим выбором при разработке программ, связанных с веб-разработкой, и послужило причиной выбора данного языка в разработке программного обеспечения для данной работы.